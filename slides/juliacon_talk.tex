\documentclass{beamer}

%\usetheme{Boadilla}

\usepackage{graphics}
\usepackage{amsmath}
\usepackage{amssymb}
\usepackage{listings}

%--------------------------------------------------
% Macros

\newcommand{\vf}[1]{\vskip0pt plus #1}

\renewcommand{\v}[1]{\mathbf{#1}} % Vector
\newcommand{\Abs}[1]{\left\lvert #1 \right\rvert}
\newcommand{\Norm}[1]{\left\lVert #1 \right\rVert}
\newcommand{\norm}[2][]{#1\lVert #2 #1\rVert}
\newcommand{\Bkt}[1]{\left( #1 \right)}
\newcommand{\Bktsq}[1]{\left[ #1 \right]}
\DeclareMathOperator*{\argmin}{arg\,min}
\newcommand{\floor}[1]{\left\lfloor #1 \right\rfloor}
\newcommand{\trans}[2]{T_{#1\leftarrow#2}}

%--------------------------------------------------
% Beamer stuff
\usefonttheme{structurebold}
\useinnertheme{default}
\useoutertheme{smoothbars}

% hacked albatross color theme
\setbeamercolor*{normal text}{fg=yellow!20!white,bg=blue!0!black}

\setbeamercolor*{example text}{fg=green!65!black}

\setbeamercolor*{structure}{fg=blue!25!white}

\setbeamercolor{palette primary}{use={structure,normal text},fg=structure.fg,bg=normal text.bg!75!black}
\setbeamercolor{palette secondary}{use={structure,normal text},fg=structure.fg,bg=normal text.bg!60!black}
\setbeamercolor{palette tertiary}{use={structure,normal text},fg=structure.fg,bg=normal text.bg!45!black}
\setbeamercolor{palette quaternary}{use={structure,normal text},fg=structure.fg,bg=normal text.bg!30!black}

\setbeamercolor*{block body}{bg=normal text.bg!90!black}
\setbeamercolor*{block body alerted}{bg=normal text.bg!90!black}
\setbeamercolor*{block body example}{bg=normal text.bg!90!black}
\setbeamercolor*{block title}{parent=structure,bg=normal text.bg!75!black}
\setbeamercolor*{block title alerted}{use={normal text,alerted text},fg=alerted text.fg!75!normal text.fg,bg=normal text.bg!75!black}
\setbeamercolor*{block title example}{use={normal text,example text},fg=example text.fg!75!normal text.fg,bg=normal text.bg!75!black}

\setbeamercolor{item projected}{fg=black}

\setbeamercolor*{sidebar}{parent=palette primary}

\setbeamercolor{palette sidebar primary}{use=normal text,fg=normal text.fg}
\setbeamercolor{palette sidebar secondary}{use=structure,fg=structure.fg}
\setbeamercolor{palette sidebar tertiary}{use=normal text,fg=normal text.fg}
\setbeamercolor{palette sidebar quaternary}{use=structure,fg=structure.fg}

\setbeamercolor*{separation line}{}
\setbeamercolor*{fine separation line}{}



%\usecolortheme{orchid} % inner color theme
%\usecolortheme{seahorse} % outer color theme
%\usecolortheme{albatross}
%\usecolortheme[rgb={0.7,0.4,0.1}]{structure}
%\usecolortheme[hsb={0.6,0.4,0.55}]{structure}
%\usecolortheme[hsb={0.6,0.4,0.45}]{structure}
%\setbeamercolor{background canvas}{bg=}

\usefonttheme[onlymath]{serif}
%\usefonttheme[stillsansseriftext,stillsansserifsmall]{serif}

\setbeamercovered{transparent}

\setbeamertemplate{navigation symbols}{}

\title[3D mapping with Geodesy]{Accurate 3D mapping with Geodesy and Proj4}
\subtitle{Point cloud SLAM via large scale least squares}
%\titlegraphic{\includegraphics[viewport=8 653 558 831,width=6cm]{figures/logos.pdf}}


\author{Chris Foster}
\institute{Fugro Roames}
\date{2016-06-19}


%===============================================================================
%===============================================================================
\begin{document}

\begin{frame}[plain]
  \titlepage
\end{frame}

%\begin{frame}
%  \frametitle{Overview}
%  \tableofcontents
%\end{frame}


%===============================================================================
\section{The problem}

\begin{frame}
  \frametitle{Roames}
  Demo of a swath survey
  \begin{itemize}
    \item
      Flight path + laser pulses for a scanline
    \item
      Power lines + trees
    \item
      Models of power lines and poles
  \end{itemize}
\end{frame}



\begin{frame}
  \frametitle{The problem}
  Mismatched survey data gives us trouble when we care about
  \begin{itemize}
    \item
      Change detection
    \item
      Feature fitting
    \item
      Object classification
  \end{itemize}
  \vf{1filll}
\end{frame}


\begin{frame}
  \frametitle{Surveying - background}
  \begin{itemize}
    \item
      Coordinate reference systems (CRSs)
      \begin{itemize}
        \item Coordinate system (typical types: cartesian, geographic, "projected")
        \item Datum - identifies the measurement procedure: originally the survey peg, then a triangulation network, now a satellite constellation
      \end{itemize}
    \item
      Tie points
      \begin{itemize}
        \item Points in two CRS which should be the same
        \item Generate automatically, should be robust to general geometry
      \end{itemize}
    \item
  \end{itemize}
  \vf{1filll}
\end{frame}


\begin{frame}
  \frametitle{Point cloud matching}
  Standard algorithm - iterative closest point

  \onslide+<1>
  \[
    \argmin_T \sum_{\v{p}\in A} \norm[\big]{ T(\v{p}) - \mu_B(T(\v{p}))}_2
  \]
\onslide+<2>
\[
J(T, f) = \frac{1}{f^\lambda} \sum_{\v{p} \in I(A,f,T,B)}
            \Big(
              N(T(\v{p})) \cdot \left[T(\v{p}) - \mu_B(T(\v{p})\right]
            \Big)^2
\]

  Interactive demo

  \begin{itemize}
    \item  Simple and effective for good initial guesses
  \end{itemize}
  (demo)
  \vf{1filll}
\end{frame}



%===============================================================================
\section{Packages}

\begin{frame}
  \frametitle{Geodesy}
\end{frame}





%===============================================================================
\section*{End}


\begin{frame}
  \frametitle{References}
  %\nocite{*}
  %\bibliographystyle{unsrt}
  %\fontsize{7pt}{7pt}
  %\bibliography{bibliography}
  \normalsize
\end{frame}



%===============================================================================
%===============================================================================
%\part{Extras}
%
%\begin{frame}
%  \frametitle{GLSL HDRI}
%  \lstset{language=C,basicstyle=\tiny}
%  \lstinputlisting[linerange=2-16]{../hdri.glsl}
%\end{frame}


%\begin{frame}
%  \frametitle{Collage Theorem}
%\end{frame}


\end{document}
